\documentclass{hw}

\title{6.437 Project, Part 1}
\author{Phil Chodrow}
\usepackage{amsmath}
\usepackage{amsthm}
\usepackage{amsfonts}
\usepackage{amssymb}

\newtheorem{thm}{Theorem}
\newtheorem{lm}{Lemma}
\newtheorem{cor}{Corrolary}
\newtheorem{clm}{Claim}

\newtheorem*{thm*}{Theorem}
\newtheorem*{lm*}{Lemma}
\newtheorem*{cor*}{Corrolary}
\newtheorem*{clm*}{Claim}

\newcommand{\norm}[1]{\left\lVert#1\right\rVert}
\newcommand\decision[2]{\underset{#2}{\overset{#1}{\gtreqless}}}
\newcommand{\argmax}{\operatornamewithlimits{argmax}}
\newcommand{\argmin}{\operatornamewithlimits{argmin}}
\newcommand\abs[1]{\left|#1\right|}
\newcommand\prob[0]{\mathbb{P}}
\newcommand\E[0]{\mathbb{E}}
\newcommand\R[0]{\mathbb{R}}

\begin{document}

%-----------------------------------------------
\problem{}
	We consider the following cdeciphering problem. Let $\mathbf{X}$ be a random variable representing the plaintext corresponding to an observed ciphertext $\mathbf{Y}$, where $\mathbf{Y} = f(\mathbf{X})$ and $f:\mathcal{A} \rightarrow \mathcal{A}$ is a permutation of the English alphabet $\mathcal{A}$, which includes lowercase letters, spaces, and periods. 

	We assume that the sequence of characters in English text can be approximated as a stationary Markov chain, where the probability of a symbol depends only on the symbol that precedes it. Formally, if we number the symbols of $\mathcal{A}$ from $1$ to $m \abs{\mathcal{A}}$, then 
	\begin{equation}
		\prob(X_k = i\;|\; X_{k-1} = j) = M_{ij}\;,
	\end{equation}
	where $\mathbf{M}$ is the transition matrix. Furthermore, let the marginal probability of each symbol be given by 
	\begin{equation}
		P_i \triangleq \prob(X_k = i), \quad i = 1, 2, \ldots, m\;,
	\end{equation}
	and let $\mathbf{P} = [P_i]$ be the vector of marginal probabilities. 

\solution
	\part
		Since $f$ is bijective by hypothesis, we can write
		\begin{align}
			p_{\mathbf{Y}|F}(\mathbf{y}|f) &= \prob(\mathbf{Y} = \mathbf{y} \;|\; F = f) \\
			&= \prob(F(\mathbf{X}) = \mathbf{y} \;|\; F = f) \\
			&= \prob(\mathbf{X} = f^{-1}(\mathbf{y})) \\
			&= (p_\mathbf{X} \circ f^{-1})(\mathbf{y})\;.
		\end{align}
	\part
		Let $\mathcal{C}$ be the space of permutations of $\mathcal{A}$. Then, $\abs{\mathcal{C}} = \abs{\mathcal{A}}!$, so that $p_F(f) = \frac{1}{\abs{\mathcal{A}}!}$ for all $f$, and we can write 
		\begin{align}
			p_{F|\mathbf{Y}}(f|\mathbf{y}) &= \frac{p_{\mathbf{Y}|F}(\mathbf{y}|f)p_F(f)}{p_\mathbf{Y}(\mathbf{y})} \\
			&= \frac{1}{\abs{\mathcal{A}}!} \frac{(p_\mathbf{X} \circ f^{-1})(\mathbf{y})}{\sum_{f \in \mathcal{C}}(p_\mathbf{X} \circ f^{-1})(\mathbf{y})}
		\end{align}
	\part
		The normalization $p_\mathbf{Y}(\mathbf{y}) = \sum_{f \in \mathcal{C}}(p_\mathbf{X} \circ f^{-1})(\mathbf{y})$ is impractical to calculate, since it involves the evaluation and summation of $\abs{A}! = 28! \approx 10^{29}$ terms.  
%-----------------------------------------------
\problem{}
	
\solution

%-----------------------------------------------
\problem{}
	
\solution


\end{document}